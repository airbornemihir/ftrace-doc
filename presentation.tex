% presentation.tex
\documentclass{beamer}

% items enclosed in square brackets are optional; explanation below
\title{ftrace for Android hackers.}
\subtitle{What ftrace is, why you should be using it, and how to get
  the most out of it.}
\author{Mihir Mehta}
\institute{
  Kernel Team\\
  Systems Core Group\\
  SRI Noida \\
  \texttt{mihir.mehta@samsung.com}
}
\date{6 January 2014}

\begin{document}

%--- the titlepage frame -------------------------%
\begin{frame}[plain]
  \titlepage
\end{frame}

%--- the presentation begins here ----------------%
\begin{frame}{Overview}
  Overview of the material.
  \begin{enumerate}
  \item Introduction to ftrace
  \item Filesystems and debugfs
  \item ftrace functionality
  \item Example
  \end{enumerate}
\end{frame}

\begin{frame}{Introduction to ftrace}
  What is ftrace anyway?
  \begin{itemize}
  \item Most commonly known as the function tracer.
  \item Idea: save timing information for the execution of kernel
    functions and export the information to userspace.
  \item Utility: This is useful to kernel developers who need to keep
    an eye on how much time particular functions are taking.
  \item Today: Expanded to include tracers for many other kinds of
    info.
  \item Output: variety of formats, including human readable output -
    immensely useful for embedded developers (that means us!)
  \end{itemize}
\end{frame}

\begin{frame}{Filesystems and debugfs - introduction}
  \begin{itemize}
  \item Linux - several different filesystems, intended for different
    purposes.
  \item Common features - encapsulated in the VFS (Virtual
    Filesystem Switch).
  \item This allows most filesystems to be mounted using the mount
    command.
  \item Our focus: debugfs.
  \item Like sysfs and procfs - intended to allow kernel developers
    to export information to userspace and get control
    instructions from userspace.
  \item Different from sysfs and procfs - both of these have
    specific purposes and their indiscriminate use for debugging is
    discouraged. Not so for debugfs.
  \end{itemize}
\end{frame}

\begin{frame}{Filesystems and debugfs - using debugfs}
  \begin{itemize}
  \item When you need debugfs in your build, you need to configure it
    into your kernel. %% What is the exact option name?
    \begin{itemize}
    \item NOTE: this options is automatically selected when you select
      any option that enables ftrace during kernel configuration. %% What is the exact option name?
    \item NOTE: Regardless of ftrace, debugfs is usually enabled by
      default, because many kernel subsystems have come to depend upon
      it. %% How default is it?
    \end{itemize}
  \item When debugfs is configured, the directory
    \texttt{/sys/kernel/debug} is created. debugfs is usually mounted
    into this directory.
  \item This mounting can be done, either by adding a line to the
    \texttt{/etc/fstab} file, or by using the mount command manually.
  \item Once this mounting is done, we can interact with ftrace, which
    stores its files in the \texttt{tracing} directory at the root of
    debugfs.
  \end{itemize}
\end{frame}

\end{document}
