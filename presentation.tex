% presentation.tex
\documentclass{beamer}

% items enclosed in square brackets are optional; explanation below
\title{ftrace for Android hackers.}
\subtitle{What ftrace is, why you should be using it, and how to get
  the most out of it.}
\author{Mihir Mehta}
\institute{
  Kernel Team\\
  Systems Core Group\\
  SRI Noida \\
  \texttt{mihir.mehta@samsung.com}
}
\date{6 January 2014}

\begin{document}

%--- the titlepage frame -------------------------%
\begin{frame}[plain]
  \titlepage
\end{frame}

%--- the presentation begins here ----------------%
\begin{frame}{Overview}
  Overview of the material.
  \begin{enumerate}
  \item Introduction to ftrace
  \item Primer: filesystems and debugfs
  \item ftrace functionality
  \item Example
  \end{enumerate}
\end{frame}

\begin{frame}{Introduction to ftrace}
  What is ftrace anyway?
  \begin{itemize}
  \item Started out as a function tracer.
  \item Idea: save timing information for the execution of kernel
    functions and export the information to userspace.
  \item Utility: This is useful to kernel developers who need to keep
    an eye on how much time particular functions are taking.
  \item Today: Expanded to include tracers for many other kinds of
    info.
  \item Output: variety of formats, including human readable output -
    immensely useful for embedded developers (that means us!)
  \end{itemize}
\end{frame}

\end{document}
